\documentclass{article}
\usepackage{amssymb}
\usepackage{hyperref}
\usepackage{fontspec}
\usepackage{amsmath} 
\usepackage{graphicx}
\usepackage{nopageno}
\usepackage{courier}
\usepackage{listings}
\usepackage{titlesec}
%\usepackage{ulem}
\usepackage{calrsfs}
\DeclareMathAlphabet{\pazocal}{OMS}{zplm}{m}{n}
\usepackage{pgfplots}
%\usepackage{CJK}
\usepackage{xeCJK}
\setmainfont{Liberation Serif}
\setmonofont{Hack}
%\setmainfont{AR PL UKai TW}
\setCJKmainfont{AR PL UKai TW}
\renewcommand{\ttdefault}{pcr}
\usepackage{mathtools}
\DeclarePairedDelimiter\ceil{\lceil}{\rceil}
\DeclarePairedDelimiter\floor{\lfloor}{\rfloor}
\XeTeXlinebreaklocale "zh"
\usepackage[a4paper, total={6in, 9in}]{geometry}
\pagestyle{plain}
\setlength{\parskip}{5pt}
\fontsize{14pt}{20pt}\selectfont
\titleformat*{\subsection}{\Large\bfseries}
\makeatletter
\newenvironment{xitemize}{%
    \@nameuse{fontitem\romannumeral\the\@itemdepth}
    \itemize
}{%
    \enditemize
}
\makeatother
\newcommand{\fontitem}{\Large}
\newcommand{\fontitemi}{\normalsize}
\newcommand{\fontitemii}{\normalsize}

\begin{document}
\begin{center}
    {\huge Reduction}
\end{center}

\section{Concept}
當我們想要證明一個語言 \(L_A\) 的性質\(P\)的時候,我們不一定需要從頭證起,而是可以藉由它和\(L_B\)的關係來證明。舉例來說,如果我們能夠證明:\\
    (i) \(L_B\)具有性質\(P\)\\
    (ii) 如果\(L_A\)不具有性質\(P\),那麼\(L_B\)就不具有性質\(P\)\\

那麼綜合這兩點,就可以得到\(L_A\)也具有性質\(P\)。\\

在這次上課中,我們關注的性質是不可被決定性,而課堂中已經證明過了 \(HALT\) 是不可被決定的,因此,對於另一個語言 \(L\) ,如果我們能夠證明「\(L\) 可被決定 \(\Rightarrow\) \(HALT\) 可被決定」,那麼事實上就證明了 \(L\) 的不可決定性。

  這就是歸約的概念。如果我們證明了「\(L_A\) 可被決定 \(\Rightarrow\) \(L_B\) 可被決定」,那麼我們說「\(L_B\) 可以歸約到 \(L_A\)」。歸約分成兩種:
    \begin{itemize}
    \item Map Reduction: \\
        證明存在一個可計算的函數 \(F\) ,使得:
        \begin{center} \(\forall w \in \Sigma*, w\in L_B \Leftrightarrow F(w) \in L_A\) \end{center}

    \item Turing Reduction:\\
        假設\(\pazocal{M}_A\) 是決定 \(L_A\) 的圖靈機,證明存在一台決定 \(L_B\) 的圖靈機 \(\pazocal{M}_B\) ,以 \(\pazocal{M}_A\) 作為其中一個子程序。
    \end{itemize}

\section{Example}
\textit{Tony} 的講義上列了還不少的例子,這邊作為範例,給出 \(L_4\) 的證明。

\(L_4:=\{\floor{\pazocal{M}} | L(\pazocal{M}) = \{a^nb^n | n\geq 0\}\}\)

Assume that \(L_4\) is decidable for contradiction. By it's decidability, there exists a Turing machine \(\pazocal{M}_4\) deciding \(L_4\)。Now we'll construct a computable procedure that decides \(HALT\)。

\noindent    (1) Take \(\floor{\pazocal{M}}\) and \(w\) as input.\\
\noindent    (2) Construct a new Turing machine \(\pazocal{M}_w\) such that:
    \begin{center}\(L(\pazocal{M}_w)=\left\{\begin{aligned}
    &\{a^nb^n|n\geq 0\}, && \text{if } \pazocal{M} \text{ accepts } w \\
    &\varnothing, && \text{otherwise}
    \end{aligned} \right.\)\end{center}
\noindent   (3) Use \(\pazocal{M}_4\) as subroutine: output the result of \(\pazocal{M}_4(\floor{\pazocal{M}_w})\).

    Before giving the construction for \(\pazocal{M}_w\), let's examine the correctness of this procedure first. That is to say, let's check that \(\pazocal{M}_4\) accepts \(\floor{\pazocal{M}_w}\) iff \((\floor{\pazocal{M}}, w) \in HALT\).

    \begin{center}\(\begin{aligned}
    \pazocal{M}_4 \text{ accepts } \floor{\pazocal{M}_w} & \Leftrightarrow L(\pazocal{M}_w) = \{a^nb^n|n \geq 0\} \\
                                               & \Leftrightarrow \pazocal{M} \text{ accepts } w \\
                                               & \Leftrightarrow (\floor{\pazocal{M}}, w) \in HALT
    \end{aligned}\)\end{center}
    So the correctness has been proved. The following is the construction for \(\pazocal{M}_w\).

\noindent(1) Input \(u\)\\
\noindent(2) Simulate the run of \(\pazocal{M}\) on \(w\).\\
\noindent(3)\\
  If \(\pazocal{M}\) accepts \(w\):\\
    If \(u\) is in \(\{a^nb^n|n\geq 0\}\):\\
      Accept \(u\).\\
    Else:\\
      Reject \(w\).\\
  Else:\\
    Reject \(w\).

    For the correctness of this construction, consider the three following cases:\\
    \begin{itemize}
        \item \(\pazocal{M}\) accepts \(w\):\\
            In such case, \(\pazocal{M}_w\) accepts \(u\) iff \(u\) is in \(\{a^nb^n | n \geq 0\}\), so \(L(\pazocal{M}_w) = \{a^nb^n | n \geq 0\}\).\\
        \item \(\pazocal{M}\) rejects \(w\):\\
            In such case, \(\pazocal{M}_w\) rejects every \(u\) in \(\Sigma^*\), so \(L(\pazocal{M}_w) = \varnothing\).\\
        \item \(\pazocal{M}\) does halt on \(w\):\\
            In such case, \(\pazocal{M}_w\) also will not halt while simulating, so \(L(\pazocal{M}_w) = \varnothing\).\\
    \end{itemize}

    As a result, \begin{center}\(L(\pazocal{M}_w)=\left\{\begin{aligned}
    &\{a^nb^n|n\geq 0\}, && \text{if } \pazocal{M} \text{ accepts } w \\
    &\varnothing, && \text{otherwise}
    \end{aligned} \right.\)\end{center}, which implies the correctness of the construction.
\end{document} 
